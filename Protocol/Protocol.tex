%***************************************************************************
% MCLab Protocol Template
%
% Embedded Computing Systems Group
% Institute of Computer Engineering
% TU Vienna
%
%---------------------------------------------------------------------------
% Vers.	Author	Date	Changes
% 1.0	bw	10.3.06	first version
% 1.1	bw	25.4.06	listing is in a different directory
% 1.2	bw	24.5.06	tutor has to be listed on title page
% 1.3	bw	16.6.06	statement about no plagiarism on title page (sign it!)
%---------------------------------------------------------------------------
% Author names:
%       bw      Bettina Weiss
%***************************************************************************

\documentclass[12pt,a4paper,titlepage,oneside]{article}
\usepackage{graphicx}            % fuer Bilder
\usepackage{listings}            % fuer Programmlistings
%\usepackage{german}              % fuer deutsche Umbrueche
\usepackage[latin1]{inputenc}    % fuer Umlaute
\usepackage{times}               % PDF files look good on screen
\usepackage{amssymb,amsmath,amsthm}
\usepackage{url}
\usepackage{enumitem}
\usepackage{fullpage}
\usepackage{style}

%***************************************************************************
% note: the template is in English, but you can use German for your
% protocol as well; in that case, remove the comment from the
% \usepackage{german} line above
%***************************************************************************


%***************************************************************************
% enter your data into the following fields
%***************************************************************************
\newcommand{\Vorname}{IhrVorname/Your First Name}
\newcommand{\Nachname}{IhrNachname/YourSurname}
\newcommand{\MatrNr}{3333333}
\newcommand{\Email}{e3333333@student.tuwien.ac.at}
\newcommand{\Part}{I}
\newcommand{\Tutor}{NameIhresTutors/NameOfYourTutor}
%***************************************************************************


%---------------------------------------------------------------------------
% include all the stuff that is the same for all protocols and students
\input ProtocolHeader.tex
%---------------------------------------------------------------------------

\begin{document}

%---------------------------------------------------------------------------
% create titlepage and table of contents
\MakeTitleAndTOC
%---------------------------------------------------------------------------


%***************************************************************************
% This is where your protocol starts
%***************************************************************************

%***************************************************************************
% remove the following lines from your own protocol file!
%***************************************************************************
\noindent
\textbf{Note:} This template is provided to show you how \LaTeX{} works and may
not contain all subsections your protocol should contain.\\
\textbf{Note:} You can, and in fact should, reuse appropriate parts
from the implementation proposal in the protocol.


%***************************************************************************
\section{Overview}
%***************************************************************************

%---------------------------------------------------------------------------
\subsection{Connections,  External Pullups/Pulldowns}
%---------------------------------------------------------------------------

\bConnections{What}{}
PORTC & external Pulldown enabled on PC0,3-5,7 \\
J12 & Connected to VCC \\
\eConnections

Write down all things we need to know to get your program running on our board.
All non-standard external connections, all switches your program needs, \dots
If we cannot figure out how we get your program running, we can not give you
points for it.


%---------------------------------------------------------------------------
\subsection{Design Decisions}
%---------------------------------------------------------------------------

Here comes the design decisions that you made during programming.

%---------------------------------------------------------------------------
\subsection{Specialities}
%---------------------------------------------------------------------------

Does you solution have something special (positive or negative)?


%***************************************************************************
\section{Main Application}
%***************************************************************************

Describe your application.


%***************************************************************************
\section{Communication}
%***************************************************************************

%---------------------------------------------------------------------------
\subsection{UDP}
%---------------------------------------------------------------------------

Explain your modules.

%***************************************************************************
\section{Physics}
%***************************************************************************

%***************************************************************************
\section{Game Inputs}
%***************************************************************************

%---------------------------------------------------------------------------
\subsection{GamePad/LSM303}
%---------------------------------------------------------------------------

%---------------------------------------------------------------------------
\subsection{PS/2}
%---------------------------------------------------------------------------


%***************************************************************************
\section{...explain your application modules ...}
%***************************************************************************

%***************************************************************************
\section{...the above were only examples}
%***************************************************************************


%***************************************************************************
\section{Problems}
%***************************************************************************

Put all problems you encountered into this section.
This is important information, which allows us to determine where there are
problems.
(Don't worry, we don't take points.)


%***************************************************************************
\section{Work}
%***************************************************************************

Estimate the work you put into solving the Application.

\begin{tabular}{|l|c|c|}\hline
	Task & Time spent \\ \hline

	reading manuals, datasheets &  5 h\\
	program design              &  5 h\\
	programming                 & 10 h\\
	debugging                   & 45 h\\
	questions, protocol         &  5 h\\ \hline

	\textbf{Total}              & 75 h\\ \hline
\end{tabular}



%***************************************************************************
\section{Theory Tasks}
%***************************************************************************


% Your answers should be brief but complete

\emph{%
In the theory task we want you to develop argumentation skills that
     allow you to reason about the problem you have to solve, and the
     solution you are designing.
Clear presentation of ideas is crucial for communication with team
     members, bosses, customers, etc.
This time we want you to prove your answers mathematically (e.g., by
     contradiction or induction).
Points are solely awarded for proper mathematical argumentation.}

A fleet of $n$ landers is orbiting moon at the equator.
The landers constitute a logical ring that serves as a communication
     network, and each participant may communicate exclusively by
     sending packets to its right neighbor and receiving from its left
     neighbor.
Each lander has a unique identifier, nodeID.
Moreover, it can check its remaining fuel.

The fleet should autonomously decide who lands first, depending on the
     remaining fuel.
For that, we assume that the lander with the most fuel serves as a
     controller.

The algorithm proceeds by sending packets that are lists of
messages. We have two kinds of messages:
\begin{itemize}
\item \texttt{(nodeID,fuel)}, where \texttt{nodeID} is the ID of a lander, and
\texttt{fuel} represents the remaining fuel of a lander.
\item \texttt{(nodeID,LAND)}, where \texttt{LAND} is a special symbol
  (indicating that the node with identifier \texttt{nodeID} should land next).
\end{itemize}

The controller sends a packet containing a list (of length 1), that
     contains the message \texttt{(nodeID,fuel)}, to its right
     neighbor.
Upon receiving a packet, a non-controller lander checks its local
     fuel, appends the message \texttt{(nodeID,fuel)} to the received
     packet, and forwards it to its right neighbor.
If the controller node receives a packet, it checks for the lowest
     fuel and sends the packet \texttt{(nodeID,LAND)} with the
     corresponding nodeID to its right neighbor (this packet contains
     one message).
If a non-controller lander receives \texttt{(nodeID,LAND)} from its
     left neighbor, it forwards it to its right neighbor.
If, in addition, \texttt{nodeID} matches its own nodeID, it lands.
We say the algorithm is terminated when  \texttt{(nodeID,LAND)}
     reaches the controller.

\QuText{%
	[2 Points] One landing:
	How many messages are sent globally (by all processes) until the algorithm
		terminates?
	Prove by induction!
	(Note: if a packet contains $\ell$ messages, by
		appending a message, one obtains a packet that contains $\ell+1$
		messages.
	We are interested in the number of \emph{messages} sent).
}


\QuText{%
	[1 Point] Improvement:
	Give a distributed algorithm with better asymptotic message complexity!
	As in the original algorithm, one of the landers with the least fuel
		should land.
	Provide instructions (what packets to send, etc.) for the controller and
		the other landers.
	As in the original, the code for the controller and the lander may be
		different.
	Prove that your algorithm is better!
}

\QuText{%
	[2 Points] Landing Fleet:
	Starting with a fleet of $n$ landers, the above process is repeated $n$
	times.
	How many messages are sent globally (by all processes) until the $n$
	iterations terminate? Prove for the original algorithm and for your
	solution!
}


%***************************************************************************
\newpage
\end{document}

