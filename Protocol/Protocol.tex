%***************************************************************************
% MCLab Protocol Template
%
% Embedded Computing Systems Group
% Institute of Computer Engineering
% TU Vienna
%
%---------------------------------------------------------------------------
% Vers.	Author	Date	Changes
% 1.0	bw	10.3.06	first version
% 1.1	bw	25.4.06	listing is in a different directory
% 1.2	bw	24.5.06	tutor has to be listed on title page
% 1.3	bw	16.6.06	statement about no plagiarism on title page (sign it!)
%---------------------------------------------------------------------------
% Author names:
%       bw      Bettina Weiss
%***************************************************************************

\documentclass[12pt,a4paper,titlepage,oneside]{article}
\usepackage{graphicx}            % fuer Bilder
\usepackage{listings}            % fuer Programmlistings
%\usepackage{german}              % fuer deutsche Umbrueche
\usepackage[latin1]{inputenc}    % fuer Umlaute
\usepackage{times}               % PDF files look good on screen
\usepackage{amssymb,amsmath,amsthm}
\usepackage{url}
\usepackage{enumitem}
\usepackage{fullpage}
\usepackage{style}
\usepackage{tikz}


%***************************************************************************
% note: the template is in English, but you can use German for your
% protocol as well; in that case, remove the comment from the
% \usepackage{german} line above
%***************************************************************************


%***************************************************************************
% enter your data into the following fields
%***************************************************************************
\newcommand{\Vorname}{Chirstoph}
\newcommand{\Nachname}{Lehr}
\newcommand{\MatrNr}{}
\newcommand{\Email}{e1525189@student.tuwien.ac.at}
\newcommand{\Part}{I}
\newcommand{\Tutor}{NameIhresTutors/NameOfYourTutor}

\newcommand{\pts}[2]{\textbf{[#1]} #2}
%***************************************************************************


%---------------------------------------------------------------------------
% include all the stuff that is the same for all protocols and students
\input ProtocolHeader.tex
%---------------------------------------------------------------------------

\begin{document}

%---------------------------------------------------------------------------
% create titlepage and table of contents
\MakeTitleAndTOC
%---------------------------------------------------------------------------


%***************************************************************************
% This is where your protocol starts
%***************************************************************************

%***************************************************************************
% remove the following lines from your own protocol file!
%***************************************************************************
\noindent
\textbf{Note:} This template is provided to show you how \LaTeX{} works and may
not contain all subsections your protocol should contain.



%***************************************************************************
\section{Overview}
%***************************************************************************

%---------------------------------------------------------------------------
\subsection{Connections,  External Pullups/Pulldowns}
%---------------------------------------------------------------------------
The Pin-Configuration is listed in table \ref{table:PinConfigurations}.
\begin{table}[]
	\begin{tabular}{|r|l|l|l|}\hline
		MCU-Pin & Used Pin-Function & Extension Board	& Purpose \\ \hline
		PF0 	& ADC0				& 					& Potentiometer \\ \hline 
		\multicolumn{4}{|c|}{FMClick} \\ \hline 
		PD0		& SCL				& SCIO 				& I2C Clock \\
		PD1 	& SDA				& SDIO				& I2C Data \\
		PD3 	&					& GPIO2				& GPIO2 for interrupts \\
		PD4 	&					& RST				& Reset pin\\\hline
		\multicolumn{4}{|c|}{Ethernet Module} \\ \hline 
		PB0 	&					& CS				& Chip select\\
		PB1 	& SCL				& SCK 				& SPI Clock\\
		PB2 	& MOSI				& MOSI				& Master out, Slave in\\
		PB3 	& MISO				& MISO 				& Master in, Slave out\\
		PB4 	&					& MP3\_RST			& Reset \\
		PD2 	& INT2				& INT 				& Interrupt Pin\\ \hline
		\multicolumn{4}{|c|}{PS2 } \\ \hline 		
		PK6 	& 					& Data 				& PS2 Data\\
		PK7 	& 	 				& Clock				& PS2 Clock\\\hline
	\end{tabular}
	\caption{Pin Configuration}
	\label{table:PinConfigurations}
\end{table}


%---------------------------------------------------------------------------
\subsection{Design Decisions}
%---------------------------------------------------------------------------

\begin{itemize}
	\item For a smoother and better handling of the volume, I implemented median calculation of the ADC values. Unfortunately I realised later, that the FMClick only uses 4 bit Volume scale instead of expected 8 bit. I decided to leave it in the code.
	\item In favour of an easier handling I chose that every time a configuration Bit was 
	changed, the first 5 registers will be transmitted. In addition I read always the first 6 register. The only 2 exceptions are during startup, where all 16 registers a read an one time 6 registers is written to.
\end{itemize}

%---------------------------------------------------------------------------
\subsection{Specialities}
%---------------------------------------------------------------------------

\begin{itemize}
	\item Most registers from the FMClick module are modelled as Unions of Bitfields. This has the advantage of easier modifications of the configuration Bits.
\end{itemize}

%***************************************************************************
\section{Main Application}
%***************************************************************************

The main application handles the interaction between the Database, FMClick, PS2 and the ADC. A Database is used to store information externally. The FMClick is used for listening to radio and retrieving information from the Radio Data Service. A PS2 Keyboard is used to
send commands to the Application. The ADC is used to set the volume of the FMClick module.

%---------------------------------------------------------------------------
\subsection{Main Screen}
%---------------------------------------------------------------------------

The main Main Screen displays the following items:
\begin{itemize}
	\item The position in the favourite list
	\item The current Date and Time, if available
	\item Frequency of the current station
	\item The name of the current station
	\item The radio text, if available
	\item A note, if available
\end{itemize}



%---------------------------------------------------------------------------
\subsection{Commands}
%---------------------------------------------------------------------------

By using a PS2-Keyboard the following commands can be executed:
\begin{itemize}
	\item n/p Switch to next/previous station
		\subitem This is done by starting a seek operation either upwards or downwards
	\item +/- Manual tune
		\subitem Step to the next/previous channel
	\item t   Tune to a specific channel 
		\subitem Done by entering a frequency
	\item l   Toggle displaying of the channel list
	\item s   Start a scan for channels
		\subitem This starts a scan over the whole spectrum 
	\item f   Add channel to favourites followed by a number between 0 and 9
		\subitem 1-9 add channel on position
		\subitem 0 to add current channel to list
	\item a   Add a note to the current station
		\subitem Allows the user to add a up to 40 character long note
\end{itemize}

%---------------------------------------------------------------------------
\subsection{Scanning}
%---------------------------------------------------------------------------

When a scan is initiated the local and the remote list is cleared and filled
up with new entries found by the scan. 

%***************************************************************************
\section{Database}
%***************************************************************************

The Database module handles the UDP with the host of Database.
The following items are stored in the on the host:
\begin{itemize}
	\item The Frequency of the station
	\item The name of the station
	\item The position in the Favourite List 
	\item The node of the station, if available
\end{itemize}

At the startup of the Application it fetches the current state from the Database. 
During runtime changed data is saved to the Database. If the station is changed and 
the new station is in the list, all stored data is retrieved from the Database.

%---------------------------------------------------------------------------
\subsection{UDP}
%---------------------------------------------------------------------------


%---------------------------------------------------------------------------
\subsection{Database Module}
%---------------------------------------------------------------------------

The Database module handles storing of data and retrieving information from the
host of the Database. The decode the data from the host, strings are tokenized and 
parsed. It supports adding and updating of stations as well as dropping all stored 
data. To get all necessary data, retrieving of a single item and the complete list 
is available.

%***************************************************************************
\section{FMCLick}
%***************************************************************************

The FMClick module is a small FM-module which is connected via I2C to the 
BigAVR6 development board. The module supports configurations for multiple 
countries and is able to decode RDS information.

%---------------------------------------------------------------------------
\subsection{I2C}
%---------------------------------------------------------------------------

For communication with the FMClick moduel the I2C bus is used. In TinyOS the
bus is modelled as a Resource. For a clean handling before each transmission the 
the Resource will be acquired and after the transmission is finished, it is released 
again.
\newline
When the FMClick module is initialized all registers are read at the beginning. 
Afterwards the oscillator is activated and the initial configuration is transmitted.
When the startup procedure is finished, every write sends the first 5 registers and 
every read reads the last 6 registers. 

%...........................................................................
\subsubsection{Register Manipulation}
%...........................................................................
For an easier modification of the registers of the chip I decided to model the
16 bit registers as bitfields. It is necessary maintain the correct order of 
the single bits. Since the high byte is transfered first, the top most entries 
have to be bits from the high byte. The first entry starts at the least significant 
bit, so the lowest bit in the high byte has to be in the first entry of the 
bitfield. As an example see Listing \ref{lst:register} which represents the register
SYSCONF1.

\begin{lstlisting}[language=C,label={lst:register}]
typedef union __sys_conf_1_t
{
	struct
	{
		const uint8_t Reserved_2    : 2; // Bits 8:9
		uint8_t AGCD                : 1; // Bit 10
		uint8_t DE                  : 1; // Bit 11
		uint8_t RDS                 : 1; // Bit 12
		const uint8_t Reserved_1    : 1; // Bit 13
		uint8_t STCIEN              : 1; // Bit 14
		uint8_t RDSIEN              : 1; // Bit 15
		uint8_t GPIO1               : 2; // Bits 0:1
		uint8_t GPIO2               : 2; // Bits 2:3
		uint8_t GPIO3               : 2; // Bits 4:5
		uint8_t BLNDADJ             : 2; // Bits 6:7
	};
	uint8_t data_bytes[2];
} sys_conf_1_t;
\end{lstlisting}

%...........................................................................
\subsubsection{Seek and Tune}
%...........................................................................
If either a seek or a tune operation is started, the application waits for 
the trigger of an external interrupt on pin PD3. When the interrupt is fired
the data registers from the FMClick module are read, if the seeking/tuning is 
completed, the respective bits are cleared and a new operation is only permitted
if the STC bit from the FMClick is cleared. This bit is polled in a 40ms period 
until it is cleared.

%---------------------------------------------------------------------------
\subsection{Radio Data Service}
%---------------------------------------------------------------------------

I implemented the following RDS Datatypes:
\begin{itemize}
	\item Radio Station 
		\subitem RDS Type 1 A and B
	\item Radio Text
		\subitem RDS Type 2 A and B
	\item Date and Time
		\subitem RDS Type 4 A
\end{itemize}

%...........................................................................
\subsubsection{Radio Station}
%...........................................................................
The name of the Station is transmitted in 4 Blocks where each Block holds 2 
characters. So after receiving of 4 Radio Station message, the Application 
should be able to construct the name of the current station. The differences
between Type A and B are not considered in the application.

%...........................................................................
\subsubsection{Radio Text}
%...........................................................................
The Radio Text is transmitted in up to 16 Blocks, with up to 64 characters.
If Type A is used, each message holds 4 characters, but Type B only uses 32 
symbols. In order to check if the retrieved message is correct, all ids have
to be received in a ascending order without gaps. If the text is shorter than
the full span, a carriage return ASCII character is sent to indicate the end 
of the string.

%...........................................................................
\subsubsection{Date and Time}
%...........................................................................
This type communicates the current time and date of the channel. The date is 
transmitted "Modified Julian Date" format. The algorithm used in the Application
can be checked by using the mjd\_test.py file. This file uses the same algorithm
but implemented in Python. 
\newline
The time is two 6 bit unsigned integers and a offset from UTC in half hours.

%***************************************************************************
\section{PS2}
%***************************************************************************

To receive the data sent from the PS2-Keyboard a Pin-Change-Interrupt on the 
clock pin is used. Everytime the clock has a falling edge the value of the 
data-line is sampled. After a complete data-frame, (11 Bits: 1 Startbit, 
8 Databits, 1 Paritybit, 1 Stopbit) the char from the received scan-code  
decrypted. To get the actual character a 2-dimensional is compared to the 
received scan-code. 

%---------------------------------------------------------------------------
\subsection{Timeout Handling}
%---------------------------------------------------------------------------

In case one Bit is drop during transmission a 1ms Second timer is used. The 
Timer is started when a new transmission starts, and stopped if all 11 Bytes 
are retrieved. If one Bit is dropped or the transmission takes to long, the 
timer resets the state machine.

%***************************************************************************
\section{Volume ADC}
%***************************************************************************

For the volume the Application uses the ADC0 connected to Port F0.Every 100 ms 
a new value from ADC0 is sampled. After 5 samples the median is calculated. If 
the median changed from the last sampling, the Volume is updated.


%***************************************************************************
\section{Problems}
%***************************************************************************
�
I started setting the channel of for the FMClick module by using the actual channel numbers.
After implementing the Database, problems showed, because now, I had to do every time a 
transformation.

%***************************************************************************
\section{Work}
%***************************************************************************

Estimate the work you put into solving the Application.

\begin{tabular}{|l|c|c|}\hline
	Task & Time spent \\ \hline

	reading manuals, datasheets & 10 h\\
	program design              &  5 h\\
	programming                 & 50 h\\
	debugging                   & 15 h\\
	questions, protocol         &  5 h\\ \hline

	\textbf{Total}              & 85 h\\ \hline
\end{tabular}



%***************************************************************************
\section{Theory Tasks}
%***************************************************************************
\input{theorie2}

%***************************************************************************
\newpage
\end{document}

